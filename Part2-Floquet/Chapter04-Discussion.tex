\chapter{Conclusion}\label{ch:floquet-discussion}

In \autoref{pt:floquet} of this thesis, we have demonstrated time-crystalline behavior shares a close relationship not only to disordered interactions but also to disorder in the driving part. The unifying concept, that is key to understanding the described phenomena, is the role of long-range order within the eigenstates (described well in e.g~\cite{vonkeyserlingkAbsoluteStabilitySpatiotemporal2016,elsePrethermalPhasesMatter2017}). In \autoref{ch:metronome-spin}, the high sensitivity of the lifetime of the whole system to the variation of the driving field on just a single site is testament to the underlying long-range order - in that case of the ground state of an Ising model. States at higher temperature do not possess long range order and thus do show neither time crystalline behavior nor lifetime enhancements (except at the edges due to topological effects). Viewing the driven pair model of \autoref{ch:rydberg-timecrystal} through this lens, it becomes apparent that no true time crystal can be expected at this level: The lowest order of the pair model just does not feature any interaction terms among the pairs and so no long-range order can be generated. Whether pair-pair interactions can change this fundamentally is questionable because there is a crucial difference to the Ising model: All eigenstates of an Ising interaction, i.e. $\sum_{ij} J_{ij}S_{z}^{(i)}S_{z}^{(j)}$, written in the symmetric sector of its $\mathbb{Z}_2$ symmetry feature long-range order because they must be superpositions of the form $\ket{\psi}\pm X\ket{\psi}$, where $X$ is the generator of the symmetry. In contrast, this is not true for eigenstates of the pair model because each $H_{pair}$ has $\ket{\pm}\propto\ket{\uparrow\downarrow}\pm \ket{\downarrow\uparrow}$ as possible eigenstates which itself are symmetric under spin-flip. Thus, the eigenstates for the pair model can fulfill the symmetry in local patches destroying the global long-range order in the process.

A consequence of this interpretation is that one learns about the structure of the eigenstates from the measurement of the time crystal protocol. An interesting application of this could be to study whether we can restore the time crystal signature by using different parameter regimes. The flexibility of the Rydberg platform allow for tuning the strength of not only Ising interactions but also of random on-site potentials which are generated through van der Waals interactions of neighboring spins~\cite{wuProgrammableOrderDisorder2024}. So one could use the existence of a time crystal as an indicator to probe the crossover between XXZ and Ising models and thus the transition between pair localization and traditional Ising-like MBL. While MBL is conjectured to be absent for long-range interaction $\alpha < 2d$ by resonance counting arguments~\cite{yaoManyBodyLocalizationDipolar2014,burinLocalizationRandomXY2015}, these estimates used coefficients of the same magnitude for Ising and hopping terms and so the Ising limit of power-law XXZ models is not studied to our knowledge.