%************************************************
\label{pt:introduction}
Classical ice cream always melts, when forgotten on a table or, as a physicist might say, it thermalizes as do all classical systems without too many conservation laws. However, in the quantum realm things might be more complicated. While it is observed that in fact most quantum systems will reach some sort of equilibrium, it is generally poorly understood whether this state can be considered thermal.  

Classical equations of motion are reversible, chaos, ergodicity, phase space configuations, statistical mechanics, thermalization is a statistical effect, gas in box with divider

Quantum equation of motion/Schrödinger equation is also reversible, unitary evolution, no direct notion of phase space with probability distribution, only quasi-probability. 
Quantum chaos, OTOCs, ergodicity
Insight: Cannot expect thermalization for all things, e.g. projectors on eigenstates of the Hamiltonian are conserved. Only local observables thermalize, e.g. magnetization.
Can show that dephasing between eigenstates leads to equilibration, but this can be exponentially slow in principle. So why do we see rapid thermalization?

ETH: all eigenstates are look locally thermal, i.e. all eigenstates expectation values are a smooth functions of the eigenstate energy. It follows, if the the initial state is reasonably sharp in energy, then the dephasing between eigenstates leads to rapid thermalization. Rapid because eigenstates close already have the same value and eigenstates far apart dephase quickly.\\
The corresponding dynamical picture: The dynamics cause rapid build-up entanglement in the system, which distributes the initially local information across the whole system, so it cannot be recovered locally. In this sense entanglement is the main driver behind thermalization.

Disordered systems, glasses break ergodicity. 

This thesis consists of two major parts: In Part 1, we focus on closed Heisenberg quantum spin system with long-range interactions, where we study localization. In Part 2, we pivot to periodically driven systems where we report 
Each part starts with a comprehensive introduction to the relevant concepts and closes with a discussion of the results including directions for future research. 
\FIXME{adapt to how summary is done then.}

Anderson localization in single-particle hopping systems. Generalization to MBL with random on-site potentials. 

%TODO think about some figure? Perhaps like visual abstract for the thesis