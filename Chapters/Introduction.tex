%************************************************
\chapter*{Introduction}
\chaptermark{Introduction} %otherwise gets it wrong? But only here!
\label{pt:introduction}
%Ice cream always melts when forgotten on a table. Or, as a physicist might say, it thermalizes, i.e. evolves towards thermal equilibrium. In every day experience, most classical systems thermalize rapidly, like e.g. aforementioned ice cream, that melts in minutes, or ripples on a pond created by a stone's impact, that settle after a few seconds. 
%There are however some rather exotic systems exhibiting anomalously slow thermalization, e.g. so-called spin glasses\cite{edwardsTheorySpinGlasses1975,binderSpinGlassesExperimental1986}. The key aspect hindering thermalization in these spin glasses are random interactions between the spins that makes it very difficult for the system to move between energetically equivalent states. Thus the disorder has essentially frozen out the dynamics and blocks the path to thermal equilibrium.
When pouring some milk into a cup of coffee, one can watch as it rather quickly distributes itself throughout the cup until a homogeneous mixture is reached. Or, as a physicist might say, the system evolves to \emph{thermal equilibrium}. The reason for this thermalization process can be stated statistically: There are just many more possible configurations where milk and coffee particles are mixed than ones where they are spatially separated, i.e. most microstates are \emph{typical}. Assuming the dynamics explore many different configurations~\footnote{More precisely, the dynamics needs to be \emph{ergodic}},
%do not preserve the separation explicitly, 
we will thus find the system in a typical microstate most of the time. Zooming out and taking a macroscopic point of view, the system appears to be in equilibrium. So, although there is still dynamics on the microscopic level, the macroscopic state appears to be stationary simply because most microstates exhibit similar macroscopic features. Notably, this equilibrium state can be described using only a \emph{few macroscopic quantities}, such as the average temperature and the ratio of milk to coffee.
Knowledge of this handful of values allows to compute all other properties of the system simply by averaging over all compatible microscopic configurations. This underpins the predictive power of \emph{statistical mechanics}. Indeed most classical systems, ranging from coffee in a cup over fridges and engines up to stars and black holes, thermalize and thus can be described by classical statistical mechanics.
% This gives enormous predictive power, since we can compute properties of the system simply by averaging over all microscopic configurations compatible with the conserved quantities.
%This simple sketch of thermalization holds true for most classical systems and was the result of the foundational work on statistical mechanics by P. Ehrenfest, L. Boltzmann, J. C. Maxwell, J. W. Gibbs and many more.
% In equilibrium, one can make predictions about a systems properties simply by averaging over all (micro-)states the system can take within the constraints of the globally conserved quantities such as total energy, temperature or milk-to-coffee ratio. 
% Indeed, most classical systems prepared out of equilibrium follow the process just sketched  and reach a state of thermal equilibrium which can be described by statistical mechanics.
% Why interesting?
% Can make predictions of equilibrium properties just be averaging states that are compatible with the few macroscopic parameters.

% Since statistical mechanics is a very powerful tool, it is important to understand its exceptions. One such exception are so-called spin-glasses, i.e. systems of classical dipoles where the interaction strength among them is randomly distributed but fixed~\cite{edwardsTheorySpinGlasses1975,sherringtonSolvableModelSpinGlass1975,binderSpinGlassesExperimental1986}. It has been shown that below a certain temperature, these systems don't reach an equilibrium state. The reason behind this failure of thermalization lies within the randomness: While there are many states of similar energy, these are separated by states of high energy. Thus, once the temperature is too low, the system cannot transition between all energetically allowed states anymore which breaks one of the core assumption of statistical mechanics called ergodicity. 

In the quantum realm, the situation turns out to be similar for many cases. For example, even isolated quantum systems initialized in a pure state are oftentimes found to thermalize rapidly. This means they reach a state where observables are consistent with a thermal description depending only on few parameters. This is quite surprising since the time evolution dictated by the Schrödinger equation preserves the purity and thermal states are usually highly mixed (i.e. a statistical mixture of many pure states). Thus an initially pure quantum state can never come close to a thermal state in state space. 
This conundrum can be alleviated by restricting to \emph{local observables}, i.e. observables that extract information from only a small subsystem such as the magnetization of a single spin. Then the measurement process effectively averages over the state of the rest of the system which is equivalent to performing a measurement on a mixed state. 
It follows that, if most of the states one averages over have the same local expectation value, then this explains the observed thermalization. This assumption is now known as \emph{eigenstate thermalization hypothesis} (ETH)~\cite{deutschQuantumStatisticalMechanics1991,srednickiChaosQuantumThermalization1994, deutschEigenstateThermalizationHypothesis2018} and is conceptually similar to the \emph{typicality} of microstates in classical systems.
%Put in simple terms, ETH guarantees the thermalization of local observables, i.e. observables that only extract information on a small subsystems such as the magnetization of a spin, by putting assertions on the system's spectral properties (cf. Sec.~\ref{sec:Thermalization-in-closed-QS} for the details). 
From a dynamical perspective, the consequences of ETH and thus the mechanism behind quantum thermalization, can be stated more intuitively: Consider some quantum spin system initialized in a product state, such that each spin has a well defined magnetization and there is no entanglement among spins. Letting the system evolve for some time, the interactions will cause entanglement to form and thus the initially local information about the initial state of each spin will be distributed throughout the whole system - hidden in the complicated correlations between all spins and utterly inaccessible to small scale measurements. In fact, 
%a local measurement on such a highly entangled system is akin to performing a measurement on a statistical mixture of systems. 
the stronger the entanglement between subsystem and rest, the more the subsystem appears to be mixed.
Thus the rapid buildup of entanglement is the main driver behind thermalization of local observables in closed quantum systems.

\FIXME{visualize the thermalization process?}


% Loosely speaking, ETH asserts that the \emph{individual eigenstates} of the system already look thermal for local observables in the sense that the expectation values match those of a thermal state (cf. Sec.~\ref{sec:Thermalization-in-closed-QS}). %remove spectral description here?

%% Now disordered systems
If interactions are the main reason behind the build-up of entanglement and thus thermalization, shouldn't every non-integrable quantum system thermalize? The story is not as simple as that! While it is true that many interacting quantum many-body systems that appear to thermalize, there are also many systems in which the nature of their late time behavior is currently hotly debated. Among these are strongly disordered spin systems, where the disorder usually takes the form of random on-site potentials and/or random inter-spin couplings. These disordered systems can arise quite naturally in many different contexts such as cold atomic gases~\cite{schreiberObservationManybodyLocalization2015,kondovDisorderInducedLocalizationStrongly2015}, color centers in diamond~\cite{kucskoCriticalThermalizationDisordered2018,martinControllingLocalThermalization2023} or generally systems with impurities~\cite{weiExploringLocalizationNuclear2018,silevitchTuningHighQNonlinear2019}.
In his seminal 1958 paper~\cite{andersonAbsenceDiffusionCertain1958}, Anderson showed that for a single excitation hopping in a lattice with random on-site potentials, all motion, and thus thermalization, arrests \emph{completely} if the randomness is sufficiently strong.
This phenomenon, now dubbed Anderson localization, was later generalized to interacting many-body spin systems under the name many-body localization (MBL)~\cite{fleishmanInteractionsAndersonTransition1980,baskoMetalinsulatorTransitionWeakly2006,gornyiInteractingElectronsDisordered2005,bauerAreaLawsManybody2013}.
While the existence of MBL was demonstrated numerically in all kinds of small systems with tens of spins, it is strongly doubted to exist in the thermodynamic limit due to the presence of rare regions with low disorder that are thought to thermalize the rest of the system. 

\FIXME{Too long, split up, more detail}


Interestingly, the presence of MBL seems to have a stabilizing effect not only in closed quantum systems but also in Floquet systems, i.e. systems that undergo a periodic driving~\cite{abaninTheoryManybodyLocalization2016,elsePrethermalPhasesMatter2017,bordiaPeriodicallyDrivingManyBody2017,elseDiscreteTimeCrystals2020a}.
Usually the driving causes the system to absorb energy and heat up, which causes it evolve into a featureless infinite temperature state. However, if the system exhibits MBL, then the energy absorption can be suppressed and the system can remain perpetually in an out-of-equilibrium state, called a time crystal.

\FIXME{more detail}

\FIXME{"In this thesis" paragraph mention the practical use of MBL to derive effective model?}

In this thesis, we study both closed and periodically driven quantum spin systems subject to some form of spatial disorder. As such this thesis consists of two major parts: 
In Part 1, we explore localization phenomena in a bond-disordered Heisenberg spin model where the disorder arises from power-law interaction between randomly positioned spins. After establishing the relevant context in Chapter~\ref{ch:concepts-thermalization}, we start in Chapter~\ref{ch:pair-localization-transition} by performing a numerical study on the model in one spatial dimension across disorder strength. We find a clear crossover from a thermalizing regime into a localized regime at sufficiently strong disorder and derive the locally (quasi-)conserved quantities, which consist of pairs of strongly interacting spin. In the following Chapter~\ref{ch:cTWA-paper}, we show how this emergent structure of the system can be exploited to compute the dynamics efficiently and accurately with a semi-classical numerical technique. Finally in Chapter~\ref{ch:experimental-pairs}, we turn to a quantum simulator based on ultra-cold Rydberg atoms, that naturally implements the type of model studied, and present two different experimental studies that show clear signatures of localization based on pairs of spins.
%In Part 1, we focus on a closed quantum spin system with long-range interactions where we study localization. Specifically, we consider the paradigmatic Heisenberg quantum spin model where the couplings between the spins arise from power-law interactions among randomly placed spins. This class of models arises in many physical systems with dipolar/electronic interactions such as e.g. cold atoms, polar molecules and NV centers in diamond.\FIXME{some citations}
In Part 2, we pivot to periodically driven systems where we report \FIXME{intro for 2nd thesis part}
Each part starts with a comprehensive introduction to the relevant concepts and closes with a discussion of the results including directions for future research.

\FIXME{adapt to how summary is done then.}

%Anderson localization in single-particle hopping systems. Generalization to MBL with random on-site potentials. 

%TODO think about some figure? Perhaps like visual abstract for the thesis