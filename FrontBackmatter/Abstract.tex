%*******************************************************
% Abstract
%*******************************************************
%\renewcommand{\abstractname}{Abstract}
\pdfbookmark[1]{Abstract}{Abstract}
% \addcontentsline{toc}{chapter}{\tocEntry{Abstract}}
\begingroup
\let\clearpage\relax
\let\cleardoublepage\relax
\let\cleardoublepage\relax

\chapter*{Abstract}
% is focus clear? -> theoretical work with experimental collaboration
% are my main results clear?

In this thesis, we study the thermalization properties in both closed and periodically driven systems subject to spatial inhomogeneities. 

In Part 1, we focus on long-range Heisenberg spin models of spatially disordered spins which can be realized experimentally by current state-of-the-art platforms.
We find numerically that the disordered couplings induced by the randomly positioned spins can lead to a many-body localized regime. Using perturbative arguments based on the real-space renormalization group, we demonstrate that the emergent quasi-conserved quantities arise from pairs of strongly interacting spins decoupling from their environment. Predictions from the resulting effective model of pairs are compared to real experimental data from a Rydberg quantum simulator for validation and are found to be highly accurate.

In Part 2, we shift our focus to periodically driven systems which are known to exhibit long-lived (meta-)stable states under certain conditions. Specifically, we consider an ordered Ising chain subject to a driving field of varying strength across different parts of the chain. We demonstrate that a configuration where the driving field has the same strength for all spins except one can dramatically prolong time-crystalline signatures. We link this behavior to the presence of approximate conservation laws stabilized by the spatial inhomogeneity. Additionally, we present preliminary results on the possibility to create a time crystal by driving the pair model derived in the first part.

\vfill
%\newpage
\begin{otherlanguage}{ngerman}
\pdfbookmark[1]{Zusammenfassung}{Zusammenfassung}
\chapter*{Zusammenfassung}

In dieser Arbeit untersuchen wir die Thermalisierungseigenschaften sowohl in isolierten als auch in periodisch getriebenen Quantensystemen mit räumlichen Inhomogenitäten.

Im ersten Teil konzentrieren wir uns auf Heisenberg-Spin-Modelle von räumlich ungeordneten Spins mit langreichweitigen Wechselwirkungen, wie sie in derzeitigen Experimenten realisiert werden können. Numerische Untersuchungen zeigen, dass durch die ungeordneten Kopplungen, die durch die zufällig positionierten Spins induzierten werden, Vielteilchen-Lokalisierung auftreten kann. Mithilfe perturbativer Argumente der Realraum-Renormierungsgruppe zeigen wir, dass die entstehenden quasi-erhaltenen Größen aus Paaren stark wechselwirkender Spins bestehen. Daraus resultiert ein effektives Modell von Paaren dessen Vohersagen mit experimentellen Daten eines Rydberg-Quantensimulators decken.

Im zweiten Teil dieser Arbeit, fokussieren wir uns auf periodisch getriebene Systeme. Diese können unter bestimmten Bedingungen von Unordnung stabilisierte, langlebige Zustände aufweisen. Konkret betrachten wir eine geordnete Ising-Kette, die einem zeitlich-periodischen Feld ausgesetzt ist. Wir zeigen, dass die Lebensdauer von zeitkristallinen Signaturen sehr empfindlich auf räumlich lokale Abweichungen des Antriebs ist. Diesen Effekt führen wir auf Quasi-Erhaltungsgrößen zurück, die durch die Abweichungen im Feld stabilisiert werden. Zusätzlich präsentieren wir vorläufige Resultate zur Frage ob das Paarmodell unter Treiben auch eine Zeitkristall-Phase ermöglicht.
\end{otherlanguage}

\endgroup

\vfill

\color{black}