%*******************************************************
% Abstract
%*******************************************************
%\renewcommand{\abstractname}{Abstract}
\pdfbookmark[1]{Abstract}{Abstract}
% \addcontentsline{toc}{chapter}{\tocEntry{Abstract}}
\begingroup
\let\clearpage\relax
\let\cleardoublepage\relax
\let\cleardoublepage\relax

\chapter*{Abstract}
% is focus clear? -> theoretical work with experimental collaboration
% are my main results clear?

In this thesis, we study the thermalization properties in both closed and periodically driven systems subject to spatial inhomogeneities. 
In Part 1, we focus on long-range Heisenberg spin models of spatially disordered spins which can be realized experimentally by current state-of-the-art platforms.
We find numerically that the disordered couplings induced by the randomly positioned spins can lead to a many-body localized regime. Using perturbative arguments based on the real-space renormalization group, we demonstrate that the emergent quasi-conserved quantities are arise from pairs of strongly interacting spins decoupling from their environment. Predictions from the resulting effective model of pairs are compared to real experimental data from a Rydberg quantum simulator for validation and are found to be highly accurate.
In Part 2, we shift our focus to periodically driven systems which are known to exhibit long-lived (meta-)stable states under certain conditions. Specifically, we consider an ordered Ising model subject to a driving field of varying strength across different parts of the chain. We demonstrate that a configuration where the driving field has the same strength for all spins except one can dramatically prolong time-crystalline signatures. We link this behavior to the presence of approximate conservation laws stabilized by the spatial inhomogeneity.

\newpage
\begin{otherlanguage}{ngerman}
\pdfbookmark[1]{Zusammenfassung}{Zusammenfassung}
\chapter*{Zusammenfassung}
Deutsche Übersetzung des vorigen Abstrakts.
\TODO{Deutsches Abstrakt}
\end{otherlanguage}

\endgroup

\vfill

\color{black}