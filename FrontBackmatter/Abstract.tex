%*******************************************************
% Abstract
%*******************************************************
%\renewcommand{\abstractname}{Abstract}
\pdfbookmark[1]{Abstract}{Abstract}
% \addcontentsline{toc}{chapter}{\tocEntry{Abstract}}
\begingroup
\let\clearpage\relax
\let\cleardoublepage\relax
\let\cleardoublepage\relax

\chapter*{Abstract}
%most QS thermalize - driven and closed alike\\
%However as of recent more and more exceptions to this rule are found and studied.
%
%In this thesis, we study the thermalization properties of in two kinds of systems: closed (Part 1) and periodically driven systems.
%In Part 1, we take motivation from current state-of-the-art experimental platforms that implement long-range dipolar interactions between spatially disordered atoms or molecules. 
%We find that this type of disorder leads to a localization transition and demonstrate that the emergent quasi-conserved quantities are due to strongly interacting pairs of spins. Subsequently, we compare predictions made by this effective model of pairs to real experimental data for validation.
%In the second Part, we pivot to periodically driven systems and demonstrate that deviation of the drive on a single site can already enhance the lifetime of the system's magnetization dramatically. 
%
%The unifying theme of this thesis, is the emergence of quasi-conserved quantities that prohibit thermalization. In Part 1, the spatial disorder leads locally to a hierarchy of energy scales which facilitates the creation of strong pairs, while in Part 2, the inhomogeneity of the drive shifts decay process further away from resonance.

% is focus clear? -> theoretical work with experimental collaboration
% are my main results clear?

In this thesis, we study the thermalization properties in both closed and periodically driven systems subject to spatial inhomogeneities. 
In Part 1, we focus on long-range Heisenberg spin models of spatially disordered spins which can be realized experimentally by current state-of-the-art platforms.
We find numerically that the disordered couplings induced by the randomly positioned spins can lead to a many-body localized regime. Using perturbative arguments based on the real-space renormalization group, we demonstrate that the emergent quasi-conserved quantities are arise from pairs of strongly interacting spins decoupling from their environment. Predictions from the resulting effective model of pairs are compared to real experimental data from a Rydberg quantum simulator for validation and are found to be highly accurate.
In Part 2, we shift our focus to periodically driven systems which are known to exhibit long-lived (meta-)stable states under certain conditions. Specifically, we consider an ordered Ising model subject to a driving field of varying strength across different parts of the chain. We demonstrate that a configuration where the driving field has the same strength for all spins except one can dramatically prolong time-crystalline signatures. We link this behavior to the presence of approximate conservation laws stabilized by the spatial inhomogeneity.

%In Part 1, we take motivation from current state-of-the-art experimental platforms using cold atoms to implement long-range Heisenberg spin models of spatially disordered spins.
%In part 2 of this thesis, we pivot to an ordered spin system subject to a periodic driving field with spatial inhomogeneity. It is demonstrated that a deviation on only a single site of an otherwise uniform drive can already enhance the lifetime of the system's magnetization dramatically. We link this behavior again to the presence of approximate conservation laws that are stabilized by the spatial inhomogeneity. % mention time crystal,  meta-stable?

\newpage
\begin{otherlanguage}{ngerman}
\pdfbookmark[1]{Zusammenfassung}{Zusammenfassung}
\chapter*{Zusammenfassung}
Deutsche Übersetzung des vorigen Abstrakts.

\end{otherlanguage}

\endgroup

\vfill

\color{black}